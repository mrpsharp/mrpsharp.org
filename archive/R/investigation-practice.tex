\documentclass{article}
\usepackage{ps}
\title{Investigation Practice}
\geometry{a5paper}

\begin{document}
\maketitle \thispagestyle{empty}

You should write up the experiment up fully.  You should include: your plan, detailing the steps you took to minimise uncertainty; a description of the apparatus involved; your data, properly presented; relevant graphs; analysis, including calculation of values where possible; conclusion.You will be marked on the following criteria:\begin{description}\item[Approach and Experimental Skill]  Your initial plan including your analysis of the underlying physics.  The care taken with your method, particularly techniques used to reduce errors (NB you must discuss these in your write-up)\item[Quality and presentation of results] A good range of data must be taken including repeated observations where necessary.  The data must be correctly displayed in tables and graphically, including best fit lines and error bars.\item[Conclusions and Evaluation] Clear conclusions must be made, backed up by a detailed discussion of the results.  Uncertainties should be taken into account and the limitations of the procedure discussed.\end{description}

PS

\end{document}
