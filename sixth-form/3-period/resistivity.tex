\documentclass[11pt]{article}
\usepackage[a4paper, margin=1cm]{geometry} 
\usepackage{graphicx}
\usepackage{amssymb}
\usepackage{changepage}

\title{Resistivity Investigation}
\author{}
\date{Due: 27 September 2013}
\pagestyle{empty}

\begin{document}
\maketitle \thispagestyle{empty}
\begin{adjustwidth}{2cm}{2cm}
You are going to carry out an investigation to measure the resistivity of a wire, taking uncertainties into account.  The basic procedure is to measure voltage and current for a variety of lengths of wire and record the results.  Then, using the cross-sectional area (calculated from the diameter), the resistance and the ratio $ \frac{l}{A} $ can be calculated.  Since the resistivity is defined in the relation: \[ R = \rho \frac{l}{A} \] a graph of $ R $ against $ \frac{l}{A} $ will have a gradient equal to the resistivity.
\end{adjustwidth}

\begin{adjustwidth}{0cm}{0.4\textwidth}
\section{Setting up the spreadsheet}
At the top of your spreadsheet, before the main data tables, should be a space for constants.  In this instance you need the average diameter you measure and the uncertainty in that diameter.

Your spreadsheet should have columns as shown below.
\emph{Note: column headings should have labels and units}
\begin{description}
    \setlength{\itemsep}{10pt}
    \item[Length] This is your measurement of the length to a correct and consistent number of decimal places.
    \item[Uncertainty in length] \emph{Note: You need to take into account factors beyond the resolution of the ruler}.
    \item[Voltage] 
    \item[Uncertainty in Voltage] 
    \item[Current] 
    \item[Uncertainty in current] 
    \item[Resistance] This is the resistance as calculated from voltage and current.
    \item[Uncertainty in resistance] This is the uncertainty in the resistance, calculated from the uncertainty in voltage and current.
    \item[\emph{l}/\emph{A}] This is the ratio discussed above.  \emph{Note: as Excel does not deal well with very large values, it is best to divide all the values by a constant power of ten and put it in the column heading.}
    \item[Uncertainty in \emph{l}/\emph{A}] This is the uncertainty in the ratio of length to area, calculated from the uncertainties in \emph{l} and \emph{A}.
\end{description}
\newpage

\section{The graph}

You should plot $R$ against $\frac{l}{A}$ as the gradient of this graph will give the resistivity ($\rho$).  Your graph should include the following features:
\begin{itemize}
    \item Points unconnected by lines;
    \item Labelled axis with units;
    \item Gridlines (to enable reading values off printouts);
    \item Error bars using the uncertainty values from your spreadsheet;
    \item A median line of best fit from Excel, with an equation.
\end{itemize}

\subsection*{A note on calculating uncertainties}
When estimating the uncertainties in calculated quantities you should remember that the \emph{fractional} uncertainties of the contributing quanties are added.  Thus the uncertainty in $R$ is given by: 
$$ \Delta R = R \left(\frac{\Delta I}{I}+\frac{\Delta V}{V} \right)$$
\section{Write-up}
You should write up this investigation to include the following sections:
\begin{description}
    \item[Method] How did you go about this experiment?  What steps did you take to reduce the uncertainty in your measurements or other variables influencing the data?
    \item[Data] Your data presented in a correctly formatted table.
    \item[Graph] Your graph, correctly formatted.
    \item[Analysis] This should include a discussion of the fit of the data to the best-fit line and an estimate of the uncertainty of your measurement of $\rho$.
\end{description}

\end{adjustwidth}
\end{document}  
