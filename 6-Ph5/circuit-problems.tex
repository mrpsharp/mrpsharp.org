\documentclass[11pt]{exam}

\usepackage[a4paper]{geometry}
\usepackage[pdfusetitle]{hyperref}

\setcounter{secnumdepth}{0}
\usepackage{titlesec}
\usepackage{fontspec}

% Set sans serif font to Arial
\setsansfont{Verdana}
% Set serifed font to Times New Roman
\setmainfont{Georgia}
% Set formats for each heading level
\titleformat*{\section}{\fontsize{14}{17}\bfseries\sffamily\centering}
\titleformat*{\subsection}{\fontsize{12}{14}\bfseries\sffamily}
\titleformat*{\subsubsection}{\fontsize{11}{13}\sffamily}

\usepackage{siunitx}
\usepackage[european,siunitx]{circuitikz}

\begin{document}
\section{Solving Circuits Part 1}
\pagestyle{empty}
\begin{questions}

\question In each of the questions below use Kirchoff's Laws to calclate the missing currents and potential differences.
\begin{parts}
    \part
    \hfill\break \begin{circuitikz}
    \draw (7,0) to[battery] (0,0)
    to [short, i=\SI{3}{\ampere}] (0,-2)
    to [R] (3,-2) to [short, i=\SI{2}{\ampere}] (3,-1.5) to [R] (6,-1.5) to [short, i=$I_1$] (6,-2)
    (3,-2) to [short, i=$I_2$] (3,-2.5) to [R] (6,-2.5) to (6,-2)
    to (7,-2) to [short, i=$I_3$] (7,0)
    ;\end{circuitikz}

    \part
    \hfill\break \begin{circuitikz}
    \draw (0,0) to[battery, l=\SI{6}{\volt}] (6,0)
    to (6,-1.5)
    to[R, l=\SI{4.5}{\volt}] (3,-1.5)
    to[R, l=$V_1$] (0,-1.5) to (0,0)
    ;\end{circuitikz}

    \part
    \hfill\break \begin{circuitikz}
    \draw (0,0) to[battery, l=\SI{9}{\volt}] (6,0)
    to (6,-2)
    to[R,l=$V_2$] (3,-2)
    to (3,-1) to[R,l=$V_1$] (1,-1) to (1,-2) to (0,-2) to (0,0)
    (3,-2) to (3,-3) to[R,l=\SI{3}{\volt}] (1,-3) to (1,-2) to (0,-2)
    ;\end{circuitikz}

    \newpage

    \part \hfill\break \begin{circuitikz}
    \draw (0,0) to[short, i=\SI{3}{\ampere}] (2,0) to [battery, l=\SI{12}{\volt}] (6,0) to[short, i=$I_1$] (8,0)
    to[R, l=$V_1$] (8,-3) to (6,-3)
    to (6,-2) to[R,l=\SI{2}{\volt}] (2,-2) to[short, i=$I_2$] (2,-3) to (0,-3) to[R, l=\SI{6}{\volt}] (0,0)
    (6,-3) to (6,-4) to[R, l=$V_2$] (2,-4) to[short, i=\SI{1}{\ampere}] (2,-3)
    ;\end{circuitikz}

     \part
    \hfill\break \begin{circuitikz}
    \draw (0,0) to[short, i=\SI{0.6}{\ampere}] (3,0)
    to[battery, l=\SI{9}{\volt}] (6,0) to [short, i=$I_1$] (9,0)
    to[R, l=$V_3$] (9,-3) to (8,-3)
    to (8,-2) to[R, l=\SI{1}{\volt}] (6,-2) to[short, i=$I_2$] (5,-2) to (5,-3) to (4,-3) to (4,-2) to[R, l=\SI{3}{\volt}] (2,-2) to[short, i=$I_3$] (1,-2) to (1,-3) to (0,-3) to[R, l=\SI{2}{\volt}] (0,0)
    (8,-3) to (8,-4) to[R, l=$V_1$] (6,-4) to[short, i=\SI{0.4}{\ampere}] (5,-4) to (5,-3)
    (4,-3) to (4,-4) to[R, l=$V_2$] (2,-4) to[short, i=\SI{0.3}{\ampere}] (1,-4) to (1,-3)

    ;\end{circuitikz}

    \part \emph{If you have finished...} Calculate the resistance of the individual resistors in (e) above and the effective resistance of the whole circuit.
\end{parts}
\newpage
\section{Solving Circuits Part 2}
\question In the circuit below, calculate:
\begin{parts}
    \part the p.d. across the \SI{3}{\kilo\ohm} resistor;
    \part the current through the \SI{5}{\kilo\ohm} resistor;
    \part the resistance of resistor $R$.

    \begin{circuitikz}
        \draw (0,0) to[battery,l=20<\volt>,  i>=6<\milli\ampere>] (9,0)
        to (9,-2) to[R, l^=3<\kilo\ohm>] (4,-2)
        to (4,-1) to[R=5<\kilo\ohm>,i=$I_1$]  (1,-1) to (1,-2) to (0,-2) to (0,0)
        (4,-2) to (4,-3) to[R, l=$R$] (1,-3) to (1,-2)
    ;\end{circuitikz}
\end{parts}

\question The circuit below includes a silicon diode which has a constant p.d. across it of \SI{0.6}{\volt}. Calculate the current through resistors $P$ and $Q$ and the cell current.

\hspace{10mm}\begin{circuitikz}
    \draw (0,0) to [R, l=\mbox{$Q=\SI{2.0}{\ohm}$}, i=$i_q$] (6,0)
    to (6,4) to[R=0.5<\ohm>] (3,4) to[battery, l=1.5<\volt>, i>=$i_p+i_q$] (0,4) to (0,0)
    (0,2) to[Do, i=$i_p$] (2,2) to[R, l=\mbox{$P=\SI{2.0}{\ohm}$}] (6,2);
\end{circuitikz}

\question Calculate the current through and potential difference across resistor $R$.

  \vspace{5mm}
  \begin{circuitikz}
        \draw (0,0) to[R=2.0<\ohm>] (0,2) to[battery, l=6.0<\volt>, i>=$I_x$] (0,4) to (5,4)
        (0,0) to (5,0) to[R=0.5<\ohm>] (5,2) to[battery, l=3.0<\volt>, i>=$I_y$] (5,4)
        (2.5,0) to[R,l=\mbox{R=\SI{10.0}{\ohm}}] (2.5,2)
        (2.5,4) to[short, i=$I_x+I_y$] (2.5,3) to (2.5,2)
    ;\end{circuitikz}
      \vspace{5mm}


\newpage

\question Five resistors having the same resistance, \SI{100}{\ohm}, are connected as shown in the circuit below. A battery of \SI{12}{\volt} and negligible internal resistance is connected across the circuit.

\begin{circuitikz}
    \draw (0,0) to[R] (6,0) to[short, -*] (6,-1)
    (0,-1) to[R, *-*] (3,-1) node[label = {A}]{} to[R] (6,-1) node[label = below:B]{} to [R] (9,-1)
    (3,-1) to (3,-2) to[R] (9,-2)
    (9,-1) to (9,-3) to[battery, l=12<\volt>] (0,-3) to (0,0);
\end{circuitikz}

\begin{parts}
    \part By considering the arrangement of resistors, determine the potential difference between the points A and B.
    \part Determine the current flowing through each of the resistors and the current through the battery.
\end{parts}

\question A transmission line can be represented by a chain of resistors. Such a chain has elements which consist of two \SI{3}{\ohm} resistors. These elements are connected to form the chain as shown in the diagram.

\begin{circuitikz}
    \draw (0,0) to[R=3<\ohm>, o-*] (3,0) to[R, l_=3<\ohm>, -*] (3,-2)
    (3,0) to[R=3<\ohm>, -*] (6,0) to[R, l_=3<\ohm>, -*] (6,-2)
    (6,0) to[R=3<\ohm>, -*] (9,0) to[R, l_=3<\ohm>, -*] (9,-2)
    (9,0) to[R=3<\ohm>, -*] (12,0) to[R, l_=3<\ohm>, -*] (12,-2);
    \draw (0,-2) to [short, o-] (12,-2);
    \draw [dashed] (3.5,0.5) -- (3.5,-2.5);
    \draw [dashed] (6.5,0.5) -- (6.5,-2.5);
\end{circuitikz}

\begin{parts}
    \part Calculate the resistance between the terminals of such a chain which consists of:
    \begin{subparts}
        \subpart 2 elements only (i.e. 4 resistors);
        \subpart 3 elements only (i.e. 6 resistors);
    \end{subparts}
    \part By calculating the resistance of an infinitely long chain of resistors, show that adding more elements to a chain which is already 3 elements long produces negligible change in resistance.
\end{parts}
\end{questions}
\end{document}
