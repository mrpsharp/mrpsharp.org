\documentclass[10pt]{article}
\usepackage{ps}
\geometry{a4paper}
\title{What do we know?}
\date{Play 2014}

\begin{document}
\maketitle\thispagestyle{empty}
 This Cultural Perspective will explore the idea of knowledge. What is it? How can we be sure we have it? Is knowledge created or discovered? Can we categorise or rank different types of knowledge? Along the way to answering these questions we will briefly alight on the work of Descartes, Russell, Popper, Ayer, Moore and more! 
\begin{table}[htdp]
\normalsize

\begin{tabular}{p{0.1\textwidth}p{0.8\textwidth}}
Week 1 & How do we define knowledge? Traditional approach: Justified true belief. What do each of these words mean? Can we be justified in believing something which isn't true? Can we know something we aren't justified in believing? \\ 
Week 2 & The logical structure of knowledge. Foundationalism in maths (geometry). Descartes attempt at epistemic foundationalism. \\ 
Week 3 & Can we refute the Sceptic? \\ 
Week 4 & Moore's ``common sense'' realism. Alternatives to foundationalism. \\ 
Week 5 & Scientific knowledge and Hume's problem.\\ 
Week 6 & Popper \& Ayer. \\ 
Week 7 & Kuhn's sociology of science. \\ 
Week 8 & Counterfactual conditionals and knowledge of other worlds. \\

\end{tabular}

\label{default}
\end{table}%

\end{document}